% Options for packages loaded elsewhere
\PassOptionsToPackage{unicode}{hyperref}
\PassOptionsToPackage{hyphens}{url}
\PassOptionsToPackage{dvipsnames,svgnames,x11names}{xcolor}
%
\documentclass[
  letterpaper,
  DIV=11,
  numbers=noendperiod]{scrartcl}

\usepackage{amsmath,amssymb}
\usepackage{iftex}
\ifPDFTeX
  \usepackage[T1]{fontenc}
  \usepackage[utf8]{inputenc}
  \usepackage{textcomp} % provide euro and other symbols
\else % if luatex or xetex
  \usepackage{unicode-math}
  \defaultfontfeatures{Scale=MatchLowercase}
  \defaultfontfeatures[\rmfamily]{Ligatures=TeX,Scale=1}
\fi
\usepackage{lmodern}
\ifPDFTeX\else  
    % xetex/luatex font selection
\fi
% Use upquote if available, for straight quotes in verbatim environments
\IfFileExists{upquote.sty}{\usepackage{upquote}}{}
\IfFileExists{microtype.sty}{% use microtype if available
  \usepackage[]{microtype}
  \UseMicrotypeSet[protrusion]{basicmath} % disable protrusion for tt fonts
}{}
\makeatletter
\@ifundefined{KOMAClassName}{% if non-KOMA class
  \IfFileExists{parskip.sty}{%
    \usepackage{parskip}
  }{% else
    \setlength{\parindent}{0pt}
    \setlength{\parskip}{6pt plus 2pt minus 1pt}}
}{% if KOMA class
  \KOMAoptions{parskip=half}}
\makeatother
\usepackage{xcolor}
\setlength{\emergencystretch}{3em} % prevent overfull lines
\setcounter{secnumdepth}{-\maxdimen} % remove section numbering
% Make \paragraph and \subparagraph free-standing
\ifx\paragraph\undefined\else
  \let\oldparagraph\paragraph
  \renewcommand{\paragraph}[1]{\oldparagraph{#1}\mbox{}}
\fi
\ifx\subparagraph\undefined\else
  \let\oldsubparagraph\subparagraph
  \renewcommand{\subparagraph}[1]{\oldsubparagraph{#1}\mbox{}}
\fi


\providecommand{\tightlist}{%
  \setlength{\itemsep}{0pt}\setlength{\parskip}{0pt}}\usepackage{longtable,booktabs,array}
\usepackage{calc} % for calculating minipage widths
% Correct order of tables after \paragraph or \subparagraph
\usepackage{etoolbox}
\makeatletter
\patchcmd\longtable{\par}{\if@noskipsec\mbox{}\fi\par}{}{}
\makeatother
% Allow footnotes in longtable head/foot
\IfFileExists{footnotehyper.sty}{\usepackage{footnotehyper}}{\usepackage{footnote}}
\makesavenoteenv{longtable}
\usepackage{graphicx}
\makeatletter
\def\maxwidth{\ifdim\Gin@nat@width>\linewidth\linewidth\else\Gin@nat@width\fi}
\def\maxheight{\ifdim\Gin@nat@height>\textheight\textheight\else\Gin@nat@height\fi}
\makeatother
% Scale images if necessary, so that they will not overflow the page
% margins by default, and it is still possible to overwrite the defaults
% using explicit options in \includegraphics[width, height, ...]{}
\setkeys{Gin}{width=\maxwidth,height=\maxheight,keepaspectratio}
% Set default figure placement to htbp
\makeatletter
\def\fps@figure{htbp}
\makeatother

\KOMAoption{captions}{tableheading}
\makeatletter
\makeatother
\makeatletter
\makeatother
\makeatletter
\@ifpackageloaded{caption}{}{\usepackage{caption}}
\AtBeginDocument{%
\ifdefined\contentsname
  \renewcommand*\contentsname{Table of contents}
\else
  \newcommand\contentsname{Table of contents}
\fi
\ifdefined\listfigurename
  \renewcommand*\listfigurename{List of Figures}
\else
  \newcommand\listfigurename{List of Figures}
\fi
\ifdefined\listtablename
  \renewcommand*\listtablename{List of Tables}
\else
  \newcommand\listtablename{List of Tables}
\fi
\ifdefined\figurename
  \renewcommand*\figurename{Figure}
\else
  \newcommand\figurename{Figure}
\fi
\ifdefined\tablename
  \renewcommand*\tablename{Table}
\else
  \newcommand\tablename{Table}
\fi
}
\@ifpackageloaded{float}{}{\usepackage{float}}
\floatstyle{ruled}
\@ifundefined{c@chapter}{\newfloat{codelisting}{h}{lop}}{\newfloat{codelisting}{h}{lop}[chapter]}
\floatname{codelisting}{Listing}
\newcommand*\listoflistings{\listof{codelisting}{List of Listings}}
\makeatother
\makeatletter
\@ifpackageloaded{caption}{}{\usepackage{caption}}
\@ifpackageloaded{subcaption}{}{\usepackage{subcaption}}
\makeatother
\makeatletter
\@ifpackageloaded{tcolorbox}{}{\usepackage[skins,breakable]{tcolorbox}}
\makeatother
\makeatletter
\@ifundefined{shadecolor}{\definecolor{shadecolor}{rgb}{.97, .97, .97}}
\makeatother
\makeatletter
\makeatother
\makeatletter
\makeatother
\ifLuaTeX
  \usepackage{selnolig}  % disable illegal ligatures
\fi
\IfFileExists{bookmark.sty}{\usepackage{bookmark}}{\usepackage{hyperref}}
\IfFileExists{xurl.sty}{\usepackage{xurl}}{} % add URL line breaks if available
\urlstyle{same} % disable monospaced font for URLs
\hypersetup{
  pdftitle={Methods},
  colorlinks=true,
  linkcolor={blue},
  filecolor={Maroon},
  citecolor={Blue},
  urlcolor={Blue},
  pdfcreator={LaTeX via pandoc}}

\title{Methods}
\author{}
\date{}

\begin{document}
\maketitle
\ifdefined\Shaded\renewenvironment{Shaded}{\begin{tcolorbox}[enhanced, breakable, frame hidden, boxrule=0pt, borderline west={3pt}{0pt}{shadecolor}, sharp corners, interior hidden]}{\end{tcolorbox}}\fi

\hypertarget{methods}{%
\subsection{Methods}\label{methods}}

Simulation studies are widely used in survey statistics to evaluate the
performance of different sampling methods and designs under various
scenarios. Testing survey weight diagnostic tests requires simulating
data under different conditions continuously. The few simulation studies
to test the performance of several diagnostic tests either used purely
simulated data or existing survey data then alter the selection
probabilities by using the survey data as the population to survey from.

The data generating process for this simulation study will utilize
existing survey data from the Bureau of Labor Statistics' Consumer
Expenditure Survey public use interview data file. The dataset for 2015
contains consumer unit characteristics, assets, and expenditure data for
consumers in the United States collected by the Census Bureau for the
Bureau of Labor Statistics by interview and diary surveys. For more
information, please visit {[}insert citation for
\url{https://www.bls.gov/cex/}{]}.

For selecting the endogenous and exogenous variables to simulate,
choosing intuitive and strong empirical relationships is key for running
the survey weight diagnostic tests. WHYY? Many diagnostic tests assume
the unweighted exogenous variables \(\vec{X}\) are a significant
predictor of the endogenous variable \(Y\).

talk about data talk about tests themselves

Constructing survey weights begin with first determining the selection
probability of an unit being selected from the population. Let
\(\pi_{Si} = \mathbb{P}(i \in S)\) be the selection probability of
surveying observation \(i\) from sample \(S\). The classic
Horvitz-Thompson (HT) estimator of the population total \(Y\) is defined
as \(\hat{Y} = \sum_{i \in S} w_{Si} y_i\) where weight
\(w_{Si} = \pi_{Si}^{-1}\). In case of non-response, the HT estimator
can be further generalized by replacing \(w_{Si}\) with
\(w_i = \pi_i^{-1}\) and
\(\pi_{i} = \mathbb{P}(i \in S, i \in F, i \in R)\) with \(F\) being the
units in the target population within the sampling frame and \(R\) being
the units in \(S\) that respond to the survey.

Since survey weights are used when there is some sort of sampling bias,
design-based inference is primarily used in diagnostic tests to
accommodate for departures from some model assumptions and complex
sampling methods. For design-based inference, consider a finite
population \(U\) with \(n\) cases. Then the population linear regression
is

\begin{equation}
    \vec{Y} = X \vec{\beta} + \vec{\epsilon},
\end{equation}

where \(\vec{Y}\) is a \(n \times 1\) vector for the response variable,
\(X\) is a \(n \times k\) matrix of explanatory \(k\) variables,
\(\vec{\beta}\) is a \(k \times 1\) vector of the coefficients, and
lastly \(\vec{\epsilon}\) is a \(n \times 1\) vector of residual errors.
\textbackslash{}

Recall that for linear regressions,
\[\text{OLS Coefficients: } B = {(X^{T} X)^{-1}  X^{'} \vec{Y} \text{ and }  \text{Residuals: } \vec{\epsilon} = \vec{Y} - X \vec{\beta}.\]

Now, suppose a sample \(S\) of \(n\) units is drawn from the population
\(U\) where \(P_i\) is the probability that the \(i\)th population unit
is selected for the sample for \(i \in \{1, ..., n\}\). Note that this
probability can be further generalized to account for nonresponse and
additional complex sampling characteristics. \textbackslash{}

Pfeffermann and Sverchov (2009) show that a design-consistent estimator
of \(\vec{\beta}\) is
\[\hat{\vec{\beta}} = (X^T W X)^{-1} X^T W \vec{Y}\] where \({\bf W}\)
is a \(n \times n\) matrix with \(w_j = P_i^{-1}\) is along the
diagonal. \(\hat{\vec{\beta}}\) is a consistent estimator of
\(\vec{\beta}\) if and only if
\[\text{Cov}(w_j, y_j \mid x_j) = 0,  \text{ or equivocally, } \hspace{4 pt} \mathbb{E}(y_j \mid x_j) = \mathbb{E}(w_j y_j \mid x_j).\]

\hypertarget{model-based-inference}{%
\subsubsection{Model-Based Inference}\label{model-based-inference}}

Take the same linear regression setup as stated in (1). In model-based
inference, additional assumptions are imposed about the error such that
it is a random variable with the properties:

\begin{enumerate}
    \item $E(\vec{\epsilon}) = 0$
    \item $E(\vec{\epsilon}^{T} \vec{\epsilon}) = \sigma^2 I$
    \item $\vec{\epsilon} \perp\!\!\!\!\perp X$
\end{enumerate}

These assumptions state that as a random variable, \(\vec{\epsilon}\)
has mean of zero, homoscedasticity and no correlation with \(X\), and is
distributed independently from the explanatory variables \(X\). These
assumptions state that OLS \(\hat{\vec{\beta}}\) is consistent, but not
necessarily unbiased such that \(E(\hat{\vec{\beta}}) \neq \vec{\beta}\)
for finite samples, though is asymptotically unbiased. Typically the
assumptions are relaxed above for model-based inferences.

\hypertarget{diagnostic-tests}{%
\subsubsection{Diagnostic Tests}\label{diagnostic-tests}}

There is a wide variety of diagnostic tests for survey weights, yet
little is know about how they adapt to certain sampling conditions and
which methods are superior in determing survey weights necessity for
analysis and inference. Nearly all survey weight diagnostic tests are
categorized into two groups: Difference in Coefficients (DC) test and
Weight Association (WA) tests. Though Bollen et al.~prove that the tests
are mathematically equivalent asymptotically, the tests vary depending
on the finite population sampling methods.

\hypertarget{difference-in-coefficients-tests}{%
\paragraph{Difference in Coefficients
Tests}\label{difference-in-coefficients-tests}}

DC tests compare the estimated coefficients of the weighted and
unweighted analyses and analyze whether the coefficient estimates
difference is statistically significantly different from zero. Take the
same regression as stated in (1) while assuming
\[E(\vec{\epsilon} \mid X) = 0 \text{ and } \text{Var}(\vec{\epsilon} \mid X) = \sigma^2 I.\]

Pfeffermann (1993) suggests an adaptation to Hausman test where with two
regressions, one with weighted covariates and the other with unweighted
covariates, and testing the difference of the coefficient estimates
using a \(\chi_{k}^2\) test with
\(k = \text{dim}(\hat{\vec{\beta}}_1 - \hat{\vec{\beta}}_2)\) degrees of
freedom. Let \(\hat{\vec{\beta}}_1\) be the weighted vector of
coefficient estimates and \(\hat{\vec{\beta}}_2\) be the nonweighted
vector of coefficient estimates.

\hypertarget{weight-association-tests}{%
\paragraph{Weight Association Tests}\label{weight-association-tests}}

The majority of WA tests regress unweighted and weighted covariates on
the dependent variable. Hausman (1978) suggests running the regression
equation
\[\vec{Y} = X \vec{\beta} + X_w \vec{\beta}_w + \vec{\epsilon}\] where
\(X_w\) is the weighted covariates of \(X\). Hausman then suggests an
\(F\)-test of \(H_0: \vec{\beta}_w = 0\) as a test of misspecification
which assumes \(\vec{\epsilon}\) is Normally distributed.



\end{document}
